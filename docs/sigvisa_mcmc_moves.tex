\documentclass{article}

\usepackage{url}
\usepackage{amsmath}
\usepackage{amsfonts}
\usepackage{cancel}
\usepackage{subcaption}
\usepackage{graphicx}


\begin{document}

\title{Sigvisa RJMCMC moves}
\author{Dave Moore}
\maketitle

%The SIGVISA state is a special case of the open-world problem described above, in which there are two kinds of unknown objects: unassociated templates, and events. Each unassociated template is specific to an observed waveform (at a particular station, channel, and frequency band), and described by a set of shape parameters, currently including arrival time, peak offset time, height, and coda decay rate. By contrast, each event exists globally, but is described by a large set of parameters, describing both the event itself (lon/lat/depth/time/mb/type) as well as the shape and wiggles of the phase arrivals for each observed waveform. So if we have 100 observed waveforms, and we want to model only P arrivals, with four parameters per arrival, ignoring wiggles, then each event will have 400 parameters, plus the six intrinsic event parameters -- though really only five since event type is a discrete param -- for a total of 405 parameters per event. 

\section{Template}

TODO

\subsection{Birth/Death}

The birth move proposes a new template. The {\em peak time} of this template (note that peak time = arrival time + peak\_offset) is sampled proportionally to the height of the current unexplained signal, i.e., the observed signal minus the current hypothesized arrivals. All other template parameters are sampled from the prior. {\em Wiggle} parameters are not currently treated specially (i.e., they are also sampled from the prior, leading to weird results in many cases), but ({\bf TODO}) should in principle also be sampled signal-conditionally, i.e. by running an FFT or similar on the unexplained signal. 

The death move kills an unassociated template at random. The template to kill is selected uniformly from all current unassociated templates. In principle, we could be smarter about this, and propose to kill templates with probability inversely proportional to their support in the signal (i.e., the ratio $p(S)/(p_U(T)p_T(S))$), so as to better focus our attention on weak templates. But that would require computing probabilities for each template during every move, effectively requiring $O(n_u)$ work as opposed to $O(1)$ work under the uniform death proposal. 

\subsection{Split/Merge}

The merge and split moves allow the combination of two adjacent templates, or the splitting of a single template into two adjacent templates. Note that since we don't in general expect there to be two adjacent templates in the actual signal, the merge move is mostly useful to allow us to combine such templates that have been created by a too-enthusiastic birth move. For example, the first template proposed by a birth move doesn't have a long enough decay time to fully explain the spike in the signal, and before it can adjust, the birth move proposes a second template in the same spot. Now the two templates together explain the signal, but neither one will easily adjust by itself to explain the whole signal, and so neither one can be easily killed. The solution is to merge them together. Since we want a merge move, detailed balance requires a corresponding split move, although I can't think of an example in which this is anything other than a necessary evil. 

The merge move begins by sorting all arrivals at a station (including both event and unassociated arrivals) by arrival time. It then proposes merging two adjacent templates $k, k+1$ with probability proportional to 
\[p_k \propto \begin{cases}
   0 & \text{if $k$ and $k+1$ are both events}\\
   t_{k+1} - t_k & \text{otherwise}.
\end{cases}\]
The merge operation consists of combining the amplitude of both templates into the shape of just one of them. First, we compute a new log-amplitude $a$ for the merged template, generated by summing the amplitudes of both templates:
\[a_M = \log\left(\exp(a_k) + \exp(a_{k+1})\right).\]
In this way the total amplitude is preserved during the merge (imagine we have two copies of the exact same template at the same time; in merging them, we now have one template with the same height as the sum of the two original templates). Then, we choose one of the two templates, $T'$, to throw out, and we keep the other, $T$. If one of the templates is an event template, we always keep that one, otherwise we make the choice uniformly at random. Finally, we assign the new amplitude $a$ to the merged template $T$, which we now refer to as $M$.

By contrast, in the split move we propose to split an arrival chosen uniformly at random from the set of all arrivals, both events and unassociated. The arrival $j$ that we split from is unchanged, except that it loses an amount of its amplitude, chosen uniformly from $a'_j \sim [0, a_j]$, to a new template $j'$ that we will create. The arrival time of $j'$ is sampled uniformly from between the two adjacent arrivals $[t_{j-1}, t_{j+1}]$; note that this is necessary since the merge move can only merge adjacent arrivals. The other shape parameters for $j'$ are sampled from the prior on unassociated templates (note that $j'$ must be an unassociated template, since the merge operation only throws away unassociated templates). 


Note the merge move requires no randomness beyond the choice of move, but the split move requires us to sample an arrival time, amplitude, and other parameters for the new template. All are sampled directly, so their randomness is accounted for in $g'(u')$, except for the amplitude which will contribute to the Jacobian determinant as follows. Let the combined amplitude be $a_M$, and suppose we sample $u'_\text{amp} \sim U(0,1)$. Then we let
\begin{align*}
h'(a_M, u'_\text{amp}) &= [\log( u'_\text{amp} \exp(a_M)), \log((1-u'_\text{amp}) \exp(a_M)) ] \\
&=  [\log( u'_\text{amp}) + a_M, \log(1-u'_\text{amp}) + a_M)]\end{align*}
giving a Jacobian of
\[\left|\frac{\partial(x, u)}{\partial(x',u')}\right| = \left|\begin{array}{cc} 1 & 1/u'_\text{amp}\\1 & -1/(1-u'_\text{amp})\end{array}\right| = \frac{1}{u'_\text{amp}} + \frac{1}{1-u'_\text{amp}} = 1/\left((1-u'\text{amp})\cdot u'_\text{amp} \right)\]

Now an acceptance ratio of 
\[\frac{\pi(x')g'(u')j_m(x')}{\pi(x)g(u)j_m(x)}\left|\frac{\partial(x', u')}{\partial(x,u)}\right|\]
\[\frac{p_T(a_M)p_S(M)}{p_T(a_T)p_U(T')p_S(T, T')} \cdot \frac{1/(t_{j+1} - t_{j-1}) \cdot p_U(T') } {1} \cdot \frac{1/(n+1)}{p_k} \cdot (1-u'_\text{amp})\cdot u'_\text{amp}\]


\section{Event}

\subsection{Birth/Death}

Event moves are somewhat complex, because they're not pure birth/death moves in the usual sense. When we birth an event, we create templates at every station in the network, for the arrivals of all modeled phases of that event (formally speaking, these templates are all just a part of the event object). However, if an arrival is actually supported by a significant spike in the signal, we expect that there will already be an unassociated template in the place we'd want to create the new phase arrival. Rather than creating a second template in the same spot, which would have the immediate effect of {\em decreasing} the signal probability (assuming the unassociated template was already a good fit), we want to leverage the work that's already gone into the unassociated template by {\em associating} it with the event. Formally this can be viewed as killing the unassociated template and using its parameters to instantiate the new event template. In this way, an event ``birth move'' is actually responsible for the deaths of many unassociated templates, and conversely an event ``death move'' will birth many new templates (note that this is what allows us to kill an event while still maintaining some explanation for the signal spikes that we had previously thought were caused by the event; without this, all event deaths would pay a huge probability penalty and be summarily rejected). 

\begin{figure}
        \centering 
        \begin{subfigure}[b]{0.3\textwidth}
                \centering   
                \includegraphics[width=\textwidth]{unass_AKBB_final}
                \caption{AKASG}
        \end{subfigure}%
        \begin{subfigure}[b]{0.3\textwidth}
                \centering   
                \includegraphics[width=\textwidth]{unass_YKR8_final}
                \caption{YKA}
        \end{subfigure}%
        \begin{subfigure}[b]{0.3\textwidth}
                \centering   
                \includegraphics[width=\textwidth]{unass_JNU_final}
                \caption{YKA}
        \end{subfigure}%

        \begin{subfigure}[b]{0.3\textwidth}
                \centering   
                \includegraphics[width=\textwidth]{unass_IL31_final}
                \caption{ILAR}
        \end{subfigure}%
        \begin{subfigure}[b]{0.3\textwidth}
                \centering   
                \includegraphics[width=\textwidth]{unass_WR1_final}
                \caption{WRA}
        \end{subfigure}%
        \begin{subfigure}[b]{0.3\textwidth}
                \centering   
                \includegraphics[width=\textwidth]{unass_FIA0_final}
                \caption{FINES}
        \end{subfigure}%

        \begin{subfigure}[b]{0.3\textwidth}
                \centering   
                \includegraphics[width=\textwidth]{unass_AS12_final}
                \caption{ASAR}
        \end{subfigure}%
        \begin{subfigure}[b]{0.3\textwidth}
                \centering   
                \includegraphics[width=\textwidth]{unass_NV01_final}
                \caption{NVAR}
        \end{subfigure}%
        %\begin{subfigure}[b]{0.3\textwidth}
        %        \centering   
        %        \includegraphics[width=\textwidth]{unass_STKA_final}
        %        \caption{STKA}
        %\end{subfigure}%
        \caption{Unassociated templates from 120 steps of MCMC at several stations recording the 2009 DPRK event. Purple indicates individual templates, green shows the sum of all templates. Note that sometimes several templates are generated to explain the same event; this is likely due to the lack of a wiggle model in the current implementation.}\label{fig:templates}
\end{figure}

\begin{figure}
\includegraphics[width=\textwidth]{hough_9stas}
\caption{Hough transform of the unassociated templates in Figure \ref{fig:templates}.}
\label{fig:hough}

\end{figure}

\subsubsection{Global Component}

{\bf Birth proposal:} The first consideration of an event birth move is the proposed event itself: location, depth, time, magnitude, and source type (natural or explosion). For simplicity, all events are proposed with natural sources of magnitude 3.5 at depth 0 (actually, this can't be strictly true, since we need our proposal distribution to be supported everywhere, in order not to screw up the death moves. but it'll be something like this...). For the location and time components, we use a Hough transform to aggregate information from unassociated templates across the network.

We use a three-dimensional Hough accumulator array, corresponding to two spatial dimensions (latitude and longitude) and a time dimension. The width of the time bins is chosen to be roughly equivalent to the time required for a P wave to travel one side length of the spatial bins, so that detections from two events occurring at the same time in the same spatial bin are likely to invert to the same or adjacent time bins. 

In the first stage, we compute a separate accumulator array for each station. For each unassociated template $i$ at station $s$, we estimate a probability that the template corresponds to a ``true'' detection: this is taken to be a logistic function of the signal-to-noise ratio for that template, so that templates with high SNR contribute fully, while those near the noise floor are given less weight. 
\[w_{s,i} = \frac{1}{1 + \exp(-\alpha \cdot (\text{SNR}(s,i) - \beta))}\]

Currently we have arbitrarily set $\alpha = 2, \beta=4$. Also for each template, for each spatial bin $(x,y)$ in the accumulator we compute the expected origin time under the assumption that the template was generated by an event occurring in the center of the spatial bin:
\[t_{s,i}(x,y) = \text{arrival\_time}(s, i) - E[\text{travel\_time}((x,y), s)].\]
This inverted origin time $t_{s,i}(x,y)$ will fall within some time bin $T$. Let $b$ denote the distance in half-time-bin-widths from $t_{s,i}(x,y)$ to the center of $T$, i.e., we will have $b=0$ if $t_{s,i}(x,y)$ is exactly at the center of $T$, and $b=1$ if it is exactly on the border with an adjacent bin $T'$. For robustness, we spread the weight of the template across the two adjacent bins, according to the (arbitrary) formula
\[w_{s,i}(x, y, T) = w_{s,i} \cdot b^3/2\]
\[w_{s,i}(x, y, T') = w_{s,i} \cdot (1 - b^3/2)\]
which splits the weight evenly when the inverted time is exactly on the border, but with a cubic falloff as the time is centered more closely within the bin. (One might imagine we'd want an exponential falloff, since the travel-time model errors are roughly Laplacian, but we also need to account for the error in having assumed the origin location is the center of the spatial bin, when in fact it might have equally well been any of the corners, so we are a bit more forgiving). 

At the level of the station accumulator, we take the weight of a space-time bin to be the {\em maximum} of all templates falling in that bin,
\[w_{s}(x,y,T) = max(\delta, \max_i w_{s,i}(x,y,T)),\]
along with an additional parameter $\delta$, interpreted as the probability of an ``missed detection'', i.e., the probability that an event could have happened in that bin even if there are no corresponding templates. The overall maximization is roughly analogous to computing the most likely association between a hypothetical event in that bin and the templates at station $s$. 

Finally, at the global level, we compute the {\em product} of the station weights, essentially treating each station weight as an independent estimate of the probability that an event occurred in a particular bin (of course, this computation is actually implemented in log space). Normalizing, we obtain a global probability distribution on space-time bins. Since our event proposal distribution needs to be continuous (in particular, we'd like it to be supported over the entire earth, since MH will only allow us to kill events in locations where a birth move would be supported), we assume a uniform distribution within each space-time bin. 

Figure \ref{fig:hough} shows the distribution resulting from a Hough transform on unassociated templates corresponding to the 2009 DPRK event (shown in Figure \ref{fig:templates}). 


{\bf Death proposal:} For the death move, we sample an event to kill with probability proportional to $p_U(T)/p_E(T)$, where $T$ denotes the set of templates generated by arrivals from event $E$, $p_U(T)$ is the probability density of those templates considered under the unassociated template models of their various stations, and $p_E(T)$ is the density of the same templates under the distribution generated by the event $E$. 

\subsubsection{Station-specific components}

Here we consider the effect of the birth/death move at individual stations. Since each arriving phase is treated independently in the birth proposal, we focus on a single arriving phase $P$; sometimes referring to the event-phase pair as $(E,P)$.

{\bf Birth proposals:} At each station, we can associate any of the unassociated templates with $(E,P)$, or we can create a new template. Let $p_{E,P}(\cdot)$ denote the distribution on a template at this station generated by phase $P$ of event $E$, and $p_U(\cdot)$ denote the distribution on unassociated templates at this station. Also let $N_U$ be the random variable corresponding to the number of unassociated templates at this station (following, probably, a Poisson distribution), with current value (at the time of the proposal) denoted by $n_U$. Then we propose to associate existing template $i$ with probability

\[a_i \propto \frac{p_{E, P}(T_i)}{p_U(T_i)}\]

versus creating a new template with probability

\[b \propto \frac{p(N_U = n_U)}{p(N_U = n_U - 1)}\]

giving a normalization constant of

\[Z_\text{birth} = \frac{p(N_U = n_U)}{p(N_U = n_U - 1)} + \sum_j  \frac{p_{E, P}(T_j)}{p_U(T_j)}.\]

For another interpretation of these probabilities, let $\alpha_i$ denote the joint distribution over all templates which are currently unassociated at this station, but associating template $i$ with the arrival $(E,P)$,
\[\alpha_i = \left(p_{E, P}(T_i)\prod_{k \ne i} p_U(T_k) \right)p(N_U = n_U - 1),\]
and let $\beta$ denote the joint probability of those same templates left entirely unassociated:
\[\beta = p(N_U = n_U)\prod_k p_U(T_k)\]

It's then straightforward to show that

\[a_i = \frac{\frac{p_{E, P}(T_i)}{p_U(T_i)}}{Z_\text{birth}} = \frac{\alpha_i}{\beta + \sum_j \alpha_j} \propto \alpha_i\]
\[b = \frac{\frac{p(N_U = n_U)}{p(N_U = n_U - 1)}}{Z_\text{birth}} = \frac{\beta}{\beta + \sum_j \alpha_j} \propto \beta,\]

So we can think of the proposal in terms of the Roman letters, corresponding to odds ratios, or Greek letters, corresponding to joint probabilities. 

One might imagine that a better proposal would also take into account the signal at the station: if there is a large unexplained spike in the signal at exactly the predicted moment of arrival, maybe that should make us more likely to propose creating a new template there, as opposed to associating some existing template. But that's more complicated to figure out, and in any case, any seriously large spikes should already have an unassociated template fit to them. 

%\begin{align*}
%a_i &= \frac{\frac{p_{E, P}(T_i)}{p_U(T_i)}}{\frac{p(N_U = n_U)}{p(N_U = n_U - 1)} + \sum_j  \frac{p_{E, P}(T_j)}{p_U(T_j)}}\\
%&= \frac{\frac{p_{E, P}(T_i)}{p_U(T_i)}}{\frac{p(N_U = n_U)}{p(N_U = n_U - 1)} + \frac{\sum_j   p_{E, P}(T_j)\prod_{k\ne j} p_U(T_k)}{\prod_k p_U(T_k)}}\\
%&= \frac{\left(p_{E, P}(T_i)\prod_{k \ne i} p_U(T_k) \right)p(N_U = n_U - 1) }{p(N_U = n_U)\prod_k p_U(T_k) + \sum_j \left(\left(p_{E, P}(T_j)\prod_{k\ne j} p_U(T_k)\right) p(N_U = n_U - 1)\right)}\\
%&= \frac{\alpha_i }{\beta + \sum_j \alpha_j}\\
%\end{align*}


{\bf Death proposals}

In an event death move, we have only two choices for each arriving phase template: we can convert the template to an unassociated template, or we can delete it entirely. We propose de-association with probability proportional to 

\[\frac{p(N_U = n_U + 1)p_U(T)}{p(N_U = n_U)}\]

%\[p_T(S)p_U(T)p(N_U = n_U + 1)\]

and deletion with probability proportional to 

\[\frac{p(S)}{p_T(S)}\]

%\[p(S) p(N_U = n_U),\]

with normalizing constant

\begin{align*}
Z_\text{death} &= \frac{p(N_U = n_U + 1)p_U(T)}{p(N_U = n_U)} + \frac{p(S)}{p_T(S)}
%\\ &= p_T(S)p_U(T) p(N_U = n_U + 1) + p(S) p(N_U = n_U)
\end{align*}

Similarly to the birth move, the death move also has an alternate interpretation in terms of probabilities (as opposed to odds ratios). 

{\bf Overall}

Suppose we propose associating template $i$. Then our acceptance ratio is
\begin{align*}
\frac{\pi(x')g'(u')j_m(x')}{\pi(x)g(u)j_m(x)} &= \frac{p_{E,P}(T_i)p(N_U = n_U-1)}{p_U(T_i)p(N_U = n_U)} \cdot \frac{1}{1} \cdot \frac{\frac{p(N_U = n_U)p_U(T_i)}{p(N_U = n_U-1)} / Z_\text{death}}{\frac{p_{E, P}(T_i)}{p_U(T_i)} / Z_\text{birth}}\\
&= \frac{Z_\text{birth}}{Z_\text{death} / p_U(T_i)}\\
&=  \frac{\frac{p(N_U = n_U)}{p(N_U = n_U - 1)} + \sum_j  \frac{p_{E, P}(T_j)}{p_U(T_j)}}{\frac{p(N_U = n_U)}{p(N_U = n_U-1)} + \frac{p(S)}{p_{T_i}(S)p_U(T_i)}}
\end{align*}

Note that $\sum_j \frac{p_{E, P}(T_j)}{p_U(T_j)}$ is the total odds that {\em some} unassociated template at this station is better explained as an $(E, P)$ phase arrival. Since we are proposing to associate an unassociated template, this is (hopefully) large. Meanwhile, $\frac{p(S)}{p_{T_i}(S)p_U(T_i)}$ is the odds of explaining the signal {\em without} the template we proposed associating, relative to with it. If this is a sizeable template, this will be quite small. 

Now consider the case where we propose creating a new template $T$. Our acceptance ratio is 

\begin{align*}
\frac{\pi(x')g'(u')j_m(x')}{\pi(x)g(u)j_m(x)} &= \frac{p_{E,P}(T)p_{T}(S)}{p(S)} \cdot \frac{1}{Q(T)} \cdot \frac{\frac{p(S)}{p_{T}(S)} / Z_\text{death}}{\frac{p(N_U = n_U)}{p(N_U = n_U - 1)} / Z_\text{birth}}\\
&= \frac{p_{E,P}(T)}{Q(T)}  \cdot \frac{Z_\text{birth}}{Z_\text{death}}\cdot \frac{p(N_U = n_U -1)}{p(N_U = n_U)}\\
&= \frac{p_{E,P}(T)}{Q(T)}  \cdot \frac{1 + \sum_j  \frac{p_{E, P}(T_j)p(N_U = n_U-1)}{p_U(T_j)p(N_U = n_U)}}{\frac{p(N_U = n_U+1)p_U(T)}{p(N_U = n_U)} + \frac{p(S)}{p_{T}(S)}}
\end{align*}

which isn't obvious to me how to simplify or interpret.

\end{document}
