% Database Schema

\chapter{Database Schema}

This chapter describes all the tables used by NET-VISA.

\section{dataset}\label{tbl-dataset}

The time ranges for the different datasets used for training,
validation, and testing.

\begin{center}
\begin{tabular}{|l|l|l|}
\hline
\bf{Name} & \bf{Datatype} & \bf{Description} \\
\hline
label & varchar(20) & one of training, validation or test \\
start\_time & number & starting time for the dataset \\
end\_time & number & ending time for the dataset \\
\hline
\end{tabular}
\end{center}

\section{static\_phaseid}\label{tbl-static-phaseid}

The list of phases used by NET-VISA.

\begin{center}
\begin{tabular}{|l|l|l|}
\hline
\bf{Name} & \bf{Datatype} & \bf{Description} \\
\hline
id & number & internal numeric identifier starting from 1\\
phase & varchar2(20) & the phase name \\
timedef & varchar2(1) & either 'd' or 'n' \\
\hline
\end{tabular}
\end{center}
If timedef is 'd', the corresponding phase is used by NET-VISA to
associate arrivals to events. Otherwise, if timedef is 'n' the phase is
not used in association, but arrivals with this phase are still
considered by NET-VISA for possible relabeling.


\section{static\_siteid}\label{tbl-static-siteid}

The list of sites whose arrivals are considered by NET-VISA.

\begin{center}
\begin{tabular}{|l|l|l|}
\hline
\bf{Name} & \bf{Datatype} & \bf{Description} \\
\hline
id & number & internal numeric identifier starting from 1\\
sta & varchar2(6) & the short identifier of the station \\
lat & float(24) & latitude \\
lon & float(24) & longitude \\
elev & float(24) & elevation \\
staname & float(24) & full name of the station \\
statype & char(2) & 'ss' (single station) or 'ar' (array station) \\
\hline
\end{tabular}
\end{center}

\section{visa\_run} \label{tbl-visa-run}

The table records the start and end times plus the results of each
inference run.

\begin{center}
\begin{tabular}{|l|l|l|}
\hline
\bf{Name} & \bf{Datatype} & \bf{Description} \\
\hline
 \underline{runid}      &     int  & run identifier\\
 run\_start  &     date & starting time of run\\
 run\_end    &     date & ending time of run\\
 numsamples &     int & number of samples per time window\\
 window     &     int & window size in seconds \\
 step       &     int & window increment in seconds \\
 seed       &     int & random number generator seed \\
 data\_start &     float(24) & start time of the dataset \\
 data\_end   &     float(24) & end time of the dataset \\
 score      &     float(24) & overall internal score \\
 f1         &     float(24) & f1 score vs LEB \\
 prec       &     float(24) & precision vs LEB \\
 recall     &     float(24) & recall vs LEB \\
 error\_avg  &     float(24) & average error of matched events vs LEB \\
 error\_sd   &     float(24) & standard deviation of matched events vs LEB\\
 descrip    &     varchar(100) & descriptive comment about the run \\
\hline
\end{tabular}
\end{center}

The ending time of the run and the various scores are filled in after
inference is over.

\section{visa\_origin}\label{tbl-visa-origin}

This table records all the predicted events.

\begin{center}
\begin{tabular}{|l|l|l|}
\hline
\bf{Name} & \bf{Datatype} & \bf{Description} \\
\hline
 \underline{runid}   & int  & run identifier from \tt{visa\_run} \\
 \underline{orid}    & int  & origin identifier \\
 lon     & float(24)  &  longitude \\
 lat     & float(24)  &  latitude \\
 depth   & float(24)  &  depth \\
 time    & float(24)  &  time \\
 mb      & float(24)  &  body wave magnitude \\
 score   & float(24)  &  internal score \\
\hline
\end{tabular}
\end{center}


\section{visa\_assoc}\label{tbl-visa-assoc}

The table records the association of events predicted by NET-VISA to the
corresponding arrivals in IDCX.

\begin{center}
\begin{tabular}{|l|l|l|}
\hline
\bf{Name} & \bf{Datatype} & \bf{Description} \\
\hline
 \underline{runid}  &  int & run identifier from \tt{visa\_run}\\
 \underline{orid}   &  int & origin identifier from \tt{visa\_origin}\\
 phase  &  varchar(20) & phase name \\
 \underline{arid}   &  int & arrival identifier from \tt{idcx.arrivals}\\
 score  &  float(24) & internal score \\
 timeres & float(24) & time residual \\
 azres   & float(24) & azimuth residual \\
 slores  & float(24) & slowness residual \\
\hline
\end{tabular}
\end{center}


